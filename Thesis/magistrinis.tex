\documentclass{VUMIFPSmagistrinis}
\usepackage{algorithmicx}
\usepackage{algorithm}
\usepackage{algpseudocode}
\usepackage{amsfonts}
\usepackage{amsmath}
\usepackage{bm}
\usepackage{caption}
\usepackage{color}
\usepackage{float}
\usepackage{graphicx}
\usepackage{listings}
\usepackage{subfig}
\usepackage{wrapfig}


\university{Vilniaus universitetas}
\faculty{Matematikos ir informatikos fakultetas}
\department{Informatikos institutas}
\papertype{Magistro darbo planas}
\title{Formalių specifikacijų taikymas projektuojant paskirstytas sistemas}
\titleineng{Applying Formal Specifications to Design Distributed Systems}
\author{Matas Savickis}

\supervisor{Karolis Petrauskas, Doc., Dr.}
\reviewer{Viačeslav Pozdniakov, Partn. Doc.}
\date{Vilnius – \the\year}

\bibliography{bibliografija}

\begin{document}
\maketitle

\tableofcontents

	\section{Įvadas}

		


 		Apache Kafka


		Apache Kafka testavimas


		Formalūs metodai


		TLA+


		TLA+ taikymo sėkmės istorijos


		Darytas researchas apie Apache Kafka

	\section{Temos aktualumas bei naujumas}
		Iki šiol, kiek mums žinoma, Apache Kafka platforma buvo specifikuota tik vieną kartą neakademiniame kontekste ir sukurta specifikacija atnešė naudos padėdama surasti sistemines klaidas.
		Dėl papildomų Apache Kafka formalių specifikacijų stokos ir praeityje pasisekusio formalaus specifikavimo manome, kad papildomi tyrimai Apache Kafka platformoje atneštu naudos surandant sisteminių klaidų arba užtikrinant, kad specifikuotoje sistemos dalyje jų nėra.
		Apache Kafka platforma turi daug naudotojų, todėl tolimesnis kokybės užtikrinimas Apache Kafka platformoje atneštų naudą.
		Sėkmingai identifikavus paskirstytų sistemų architektūrinius šablonus būtų galima įrodyti korektiškumą daug didesniai pasikirstytų sistemų aibei ir šio darbo rezultatais būtų galima vadovautis kuriant atitinkamas paskirstytas sistemas. 
	
	\section{Darbo tikslas}
		Pagerinti Apache Kafka platformos kokybę surandant sisteminių klaidų arba įrodant, kad sukurtoje specifikacijoje klaidų nėra.
		Įrodyti didenės pasikirstytų sistemų aibės korektiškumas specifikuojant architektūrinius šablonus.
	
	\section{Uždaviniai}
		\begin{enumerate}
			\item{Išnagrinėti literatūrą susijusią su formaliais metodais, TLA+ specifikavimo kalba bei Apache Kafka patlforma.}
			\item{Išskirti Apache Kafka platformos modulius, kurie bus formaliai specifikuoti.}
			\item{Specifikuoti išskirtas Apache Kafka platformos dalis naudojant TLA+ specifikavimo kalbą.}
			\item{Įvertinti sukurtos formalios specifikacijos korektiškumą.}
			\item{Įvertinti, ar pagal sukurtą specifikaciją Apache Kafka platforma veikia korektiškai.}
			\item{Esant poreikiui įrodyti specifikacijos teoremas.}
			\item{Patikrinti ar Apache Kafka implementaciją atitinka specifikaciją.}
			\item{Apache Kafka platformoje surasti architektūrinius šablonus, kurie yra taikoki ir kituose paskirstytuose sistemose ir juos specifikuoti.}
		\end{enumerate}
	
	\section{Laukiami rezultatai}
		\begin{enumerate}
			\item{Pasirinktų Apache Kafka modulių specifikacija.}
			\item{Įrodymas apie specifikacijos korektiškumą.}
			\item{Apache Kafka parašytos specifikacijos ir implementacijos sutapimo įvertinimas.}
			\item{Išskirti ir specifikuoti paskirstytų sistemų šablonai taikomi kitose platformose.}
		\end{enumerate}
	
	\printbibliography[heading=bibintoc] 


\appendix


\end{document}
