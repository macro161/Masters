\documentclass{VUMIFPSmagistrinis}
\usepackage{algorithmicx}
\usepackage{algorithm}
\usepackage{algpseudocode}
\usepackage{amsfonts}
\usepackage{amsmath}
\usepackage{bm}
\usepackage{caption}
\usepackage{color}
\usepackage{float}
\usepackage{graphicx}
\usepackage{listings}
\usepackage{subfig}
\usepackage{wrapfig}


\university{Vilniaus universitetas}
\faculty{Matematikos ir informatikos fakultetas}
\department{Informatikos institutas}
\papertype{Magistro baigiamasis darbas}
\title{Formalių specifikacijų taikymas projektuojant paskirstytas sistemas}
\titleineng{Applying Formal Specifications to Design Distributed Systems}
\author{Matas Savickis}

\supervisor{Karolis Petrauskas, Doc., Dr.}
\reviewer{Viačeslav Pozdniakov, Partn. Doc.}
\date{Vilnius – \the\year}

\bibliography{bibliografija}

\begin{document}
\maketitle

%% Padėkų skyrius
% \sectionnonumnocontent{}
% \vspace{7cm}
% \begin{center}
%     Padėkos asmenims ir/ar organizacijoms
% \end{center}

\sectionnonumnocontent{Santrauka}
	

% Nurodomi iki 5 svarbiausių temos raktinių žodžių (terminų).
% Vienas terminas gali susidėti iš kelių žodžių.
\raktiniaizodziai{Apache Kafka, TLA+, Formalios specifikacijos, Paskirstytos sistemos}   


\sectionnonumnocontent{Summary}


\tableofcontents

\sectionnonum{Įvadas}

\section{Temos aktualumas bei naujumas}

\section{Darbo tikslas}

\section{Uždaviniai}

\section{Laukiami rezultatai}


\printbibliography[heading=bibintoc] 
\sectionnonum{Santrumpos}

\appendix


\end{document}
